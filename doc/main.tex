% ===============================================
% Template Info/Credits
%
% Proof template for real analysis
% Diana Davis
% Creative Commons 4.0
% https://www.overleaf.com/latex/templates/proof-template-for-real-analysis/ztkshxkdstfk
% ===============================================

\documentclass{article}

\usepackage[margin=1in]{geometry}
\usepackage{amsmath,amsthm,amssymb,hyperref}

\newcommand{\R}{\mathbf{R}}
\newcommand{\Z}{\mathbf{Z}}
\newcommand{\N}{\mathbf{N}}
\newcommand{\Q}{\mathbf{Q}}

\newenvironment{theorem}[2][Theorem]{\begin{trivlist}
\item[\hskip \labelsep {\bfseries #1}\hskip \labelsep {\bfseries #2.}]}{\end{trivlist}}
\newenvironment{lemma}[2][Lemma]{\begin{trivlist}
\item[\hskip \labelsep {\bfseries #1}\hskip \labelsep {\bfseries #2.}]}{\end{trivlist}}
\newenvironment{exercise}[2][Exercise]{\begin{trivlist}
\item[\hskip \labelsep {\bfseries #1}\hskip \labelsep {\bfseries #2.}]}{\end{trivlist}}
\newenvironment{problem}[2][Problem]{\begin{trivlist}
\item[\hskip \labelsep {\bfseries #1}\hskip \labelsep {\bfseries #2.}]}{\end{trivlist}}
\newenvironment{question}[2][Question]{\begin{trivlist}
\item[\hskip \labelsep {\bfseries #1}\hskip \labelsep {\bfseries #2.}]}{\end{trivlist}}
\newenvironment{corollary}[2][Corollary]{\begin{trivlist}
\item[\hskip \labelsep {\bfseries #1}\hskip \labelsep {\bfseries #2.}]}{\end{trivlist}}

\newenvironment{solution}{\begin{proof}[Solution]}{\end{proof}}

\begin{document}

% ------------------------------------------ %
%                 START HERE                  %
% ------------------------------------------ %

\title{Jacobian of n-DOF End-Effectors} % Replace with appropriate title
\author{johnlime} % Replace "Author's Name" with your name

\maketitle

% -----------------------------------------------------
% The following two environments (theorem, proof) are
% where you will enter the statement and proof of your
% first problem for this assignment.
%
% In the theorem environment, you can replace the word
% "theorem" in the \begin and \end commands with
% "exercise", "problem", "lemma", etc., depending on
% what you are submitting.
% -----------------------------------------------------

\begin{problem}{}
  Obtain the Jacobian for the end effector of a 2-dimensional $n$-DOF arm (\verb|IKPigeon|).
\end{problem}

\begin{proof}
  Let the position of the end effector of the arm be $(x, y)$.

  \begin{align}
    x &= - \sum^n_i C_i \cos \left( \sum^i_j \theta_j \right) \\
    y &= - \sum^n_i C_i \sin \left( \sum^i_j \theta_j \right)
  \end{align}
  where $C_i$ is the length of the limb assigned to the $i$th joint, and $/theta_i$ is its angle.

  The Jacobian matrix $J$ is composed of the gradient of each of the positions relative to $\theta$ for all joints.

  \begin{equation}
    J =
    \begin{bmatrix}
      \frac {\partial x} {\partial \theta_0} & \frac {\partial x} {\partial \theta_1} & ... & \frac {\partial x} {\partial \theta_n} \\
      \frac {\partial y} {\partial \theta_0} & \frac {\partial y} {\partial \theta_1} & ... & \frac {\partial y} {\partial \theta_n} \\
    \end{bmatrix}
  \end{equation}

  $\frac {\partial x} {\partial \theta_l}$ and $\frac {\partial y} {\partial \theta_l}$ must be calculated for each joint $l$.

  We first calculate the former gradient.
  \begin{align*}
    \frac {\partial x} {\partial \theta_l}
      &= \frac {\partial} {\partial \theta_l}
        \left(
          - \sum^{l-1}_i C_i \cos \left( \sum^i_j \theta_j \right)
          - \sum^n_l C_i \cos \left( \theta_l + \sum^i_{j \ne l} \theta_j \right)
        \right) \\
      &= \frac {\partial} {\partial \theta_l}
        \left(
          - \sum^n_l C_i \cos \left( \theta_l + \sum^i_{j \ne l} \theta_j \right)
        \right)
      && \because \frac {\partial x} {\partial \theta_l} \cos \left( \sum^i_j \theta_j \right) = 0 \thickspace \text{when} \thickspace i < l \\
      &= - \frac {\partial} {\partial \theta_l}
        \left(
          \sum^n_l C_i
            \left(
              \cos \theta_l \cos \left( \sum^i_{j \ne l} \theta_j \right) +
              \sin \theta_l \sin \left( \sum^i_{j \ne l} \theta_j \right)
            \right)
        \right) \\
      &= - \sum^n_l C_i \frac {\partial} {\partial \theta_l}
        \left(
          \cos \theta_l \cos \left( \sum^i_{j \ne l} \theta_j \right) +
          \sin \theta_l \sin \left( \sum^i_{j \ne l} \theta_j \right)
        \right) \\
      &=
      - \sum^n_l C_i
      \left(
        \left(
          \left(
            \frac {\partial} {\partial \theta_l} \cos \theta_l
          \right) \cos \left( \sum^i_{j \ne l} \theta_j \right)
          +
          \cos \theta_l
          \left(
            \frac {\partial} {\partial \theta_l}
            \cos \left( \sum^i_{j \ne l} \theta_j \right)
          \right)
        \right) \right. \\
        &+ \left.
        \left(
          \left(
            \frac {\partial} {\partial \theta_l} \sin \theta_l
          \right) \sin \left( \sum^i_{j \ne l} \theta_j \right)
          +
          \sin \theta_l
          \left(
            \frac {\partial} {\partial \theta_l}
            \sin \left( \sum^i_{j \ne l} \theta_j \right)
          \right)
        \right)
      \right)
  \end{align*}

  Since
  $\frac {\partial} {\partial \theta_l}
  \cos \left( \sum^i_{j \ne l} \theta_j \right) = 0$
  and
  $\frac {\partial} {\partial \theta_l}
  \sin \left( \sum^i_{j \ne l} \theta_j \right) = 0$,

  \begin{align*}
    \frac {\partial x} {\partial \theta_l}
      &=
      - \sum^n_l C_i
      \left(
        \left(
          \left(
            \frac {\partial} {\partial \theta_l} \cos \theta_l
          \right) \cos \left( \sum^i_{j \ne l} \theta_j \right)
        \right)
        +
        \left(
          \left(
            \frac {\partial} {\partial \theta_l} \sin \theta_l
          \right) \sin \left( \sum^i_{j \ne l} \theta_j \right)
        \right)
      \right) \\
      &=
      - \sum^n_l C_i
      \left(
        \left(
          - \sin \theta_l
        \right) \cos \left( \sum^i_{j \ne l} \theta_j \right)
        +
        \left(
          \cos \theta_l
        \right) \sin \left( \sum^i_{j \ne l} \theta_j \right)
      \right) \\
      &=
      \sum^n_l C_i
      \left(
        \sin \theta_l \cos \left( \sum^i_{j \ne l} \theta_j \right)
        -
        \cos \theta_l \sin \left( \sum^i_{j \ne l} \theta_j \right)
      \right)
  \end{align*}

  Therefore, the former gradient $\frac {\partial x} {\partial \theta_l}$ can be expressed as the following.

  \begin{equation}
    \frac {\partial x} {\partial \theta_l} =
      \sum^n_l C_i
      \left(
        \sin \theta_l \cos \left( \sum^i_{j \ne l} \theta_j \right)
        -
        \cos \theta_l \sin \left( \sum^i_{j \ne l} \theta_j \right)
      \right)
  \end{equation}

  Similarly, the latter gradient $\frac {\partial y} {\partial \theta_l}$ can be expressed as the following.

  \begin{equation}
    \frac {\partial y} {\partial \theta_l} =
      \sum^n_l C_i
      \left(
        \sin \theta_l \sin \left( \sum^i_{j \ne l} \theta_j \right)
        -
        \cos \theta_l \cos \left( \sum^i_{j \ne l} \theta_j \right)
      \right)
  \end{equation}
\end{proof}


% ---------------------------------------------------
% Anything after the \end{document} will be ignored by the typesetting.
% ----------------------------------------------------

\end{document}
