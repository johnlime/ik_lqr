% ===============================================
% Template Info/Credits
%
% Proof template for real analysis
% Diana Davis
% Creative Commons 4.0
% https://www.overleaf.com/latex/templates/proof-template-for-real-analysis/ztkshxkdstfk
% ===============================================

\documentclass{article}

\usepackage[margin=1in]{geometry}
\usepackage{amsmath,amsthm,amssymb,hyperref}

\newcommand{\R}{\mathbf{R}}
\newcommand{\Z}{\mathbf{Z}}
\newcommand{\N}{\mathbf{N}}
\newcommand{\Q}{\mathbf{Q}}

\newenvironment{theorem}[2][Theorem]{\begin{trivlist}
\item[\hskip \labelsep {\bfseries #1}\hskip \labelsep {\bfseries #2.}]}{\end{trivlist}}
\newenvironment{lemma}[2][Lemma]{\begin{trivlist}
\item[\hskip \labelsep {\bfseries #1}\hskip \labelsep {\bfseries #2.}]}{\end{trivlist}}
\newenvironment{exercise}[2][Exercise]{\begin{trivlist}
\item[\hskip \labelsep {\bfseries #1}\hskip \labelsep {\bfseries #2.}]}{\end{trivlist}}
\newenvironment{problem}[2][Problem]{\begin{trivlist}
\item[\hskip \labelsep {\bfseries #1}\hskip \labelsep {\bfseries #2.}]}{\end{trivlist}}
\newenvironment{question}[2][Question]{\begin{trivlist}
\item[\hskip \labelsep {\bfseries #1}\hskip \labelsep {\bfseries #2.}]}{\end{trivlist}}
\newenvironment{corollary}[2][Corollary]{\begin{trivlist}
\item[\hskip \labelsep {\bfseries #1}\hskip \labelsep {\bfseries #2.}]}{\end{trivlist}}

\newenvironment{solution}{\begin{proof}[Solution]}{\end{proof}}

\begin{document}

% ------------------------------------------ %
%                 START HERE                  %
% ------------------------------------------ %

\title{Jacobian of n-Joint End-Effector} % Replace with appropriate title
\author{johnlime} % Replace "Author's Name" with your name

\maketitle

% -----------------------------------------------------
% The following two environments (theorem, proof) are
% where you will enter the statement and proof of your
% first problem for this assignment.
%
% In the theorem environment, you can replace the word
% "theorem" in the \begin and \end commands with
% "exercise", "problem", "lemma", etc., depending on
% what you are submitting.
% -----------------------------------------------------

\begin{theorem}{(Page 4 $\#$ 6)}
For any metric space $E$, the entire space $E$ is an open set.
\end{theorem}

\begin{proof}
Replace this text with the details of your proof. Mathematical symbols go between dollar signs, like this: $E$, \emph{not} in plain text like this: E. If there are more symbols you want, such as $\in$, you can find them at\\ \url{http://detexify.kirelabs.org/classify.html}. If you want to set expressions or equations out from the text so that they are more readable (great idea!), put them between double dollar signs: $$|x-x_0| < \delta \implies |f(x)-f(x_0)| < \epsilon.$$
The small square marks the end of the proof.
\end{proof}


% ---------------------------------------------------
% Anything after the \end{document} will be ignored by the typesetting.
% ----------------------------------------------------

\end{document}
